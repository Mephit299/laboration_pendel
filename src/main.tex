%! Author = levi.hogdal
%! Date = 2023-10-03

% Preamble
\documentclass[11pt]{article}

% Packages
\usepackage{amsmath}
\usepackage{graphicx}
\graphicspath{ {./images/} }

\title{Labrapprt \\ \small Fysik 2}
\author{Levi Högdal }
\date{\today}

\begin{document}

    \begin{titlepage}
        \begin{center}
            \vspace*{1cm}

            \Huge
            \textbf{Laboration 2}

            \vspace{0.5cm}
            \LARGE
            Pendel

            \vspace{1.5cm}

            \textbf{Levi Högdal}

            \vfill


            Fysik 2

            \vspace{0.8cm}

            \includegraphics[width=0.4\textwidth]{../images/NTI Gymnasiet_Symbol_print_svart.png}

            \Large
            Teknikprogrammet\\
            NTI Gymnasiet\\
            Umeå\\
            \today

        \end{center}
    \end{titlepage}
    \section{Syfte och frågeställning}
    Vi ska matematiskt bestämma med vart en kula kommer landa om man skjuter den ut ur en kulkanon och sen kolla om våra beräkningar stämmer med vart kulan hamnar i verklighet.
    \subsection{Material och metod}
    \paragraph{Material:}
    \begin{itemize}
        \item En kulkanon
        \item En kula
        \item En linjal/måtband
        \item En låda där kulan ska landa
    \end{itemize}
    \subsection{Metod}


    \subsection{Resultat}

    \begin{itemize}
        \item Utgångshastighet = $5.03 m/s$
        \item Hår långt kullan har farit horisontellt när den landade på bänken = 2.23 m
        \item Hår långt kullan har farit horisontellt när den landade på golvet = 2.59 m
    \end{itemize}
    \subsection{Analys}

    \section{Diskussion}



\end{document}